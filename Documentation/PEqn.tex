\documentclass[a4paper, 12 pt, fleqn]{article}
\title{OpenFOAM\textsuperscript{\textregistered} v.4.0 \\ {\tt RectingTwoPhaseEulerFoam}}
\author{Simone Colucci, INGV-Pisa}
\date{April 2016.}
\usepackage{syntonly}
%\syntaxonly

\usepackage{array}
\usepackage{bm}
\usepackage{bmpsize}
\usepackage{booktabs}
%\usepackage[font = scriptsize, format=hang, labelfont = {sf, bf}]{caption}
\usepackage{chngpage}
\usepackage{enumitem}
\usepackage{eurosym}
\usepackage{fancyhdr}
\usepackage{float}
\usepackage{footmisc}
\usepackage{indentfirst}
\usepackage{layaureo}
\usepackage{listings}
\usepackage{longtable}
\usepackage{microtype}
\usepackage{mparhack}
\usepackage{showlabels}
%\usepackage{siunitx}
%\usepackage{SIunits}
\usepackage{subfig}
\usepackage{tabularx}
\usepackage{tipa}
%
\usepackage{tikz}
\usetikzlibrary{shapes,arrows}
%
\usepackage{url}
%\usepackage[english]{varioref}
\usepackage{xcolor}
\usepackage{wrapfig}
\usepackage{cancel} % serve per cancellare termini nelle formule
%\renewcommand{\CancelColor}{\blue}




\usepackage[latin1]{inputenc}
\usepackage{lmodern}
\usepackage[T1]{fontenc}
\usepackage[english]{babel}

\usepackage{graphicx}

\usepackage{textcomp}
\usepackage{verbatim}
\usepackage{latexsym}
\usepackage{amsmath, amssymb, amsthm, eucal, multicol}
\usepackage{esdiff}
\usepackage{amsfonts}

\usepackage[pagebackref]{hyperref}
\usepackage[hyperpageref]{backref}
\usepackage{listings}
\lstset{language=C++,captionpos=b,tabsize=3,frame=lines,keywordstyle=\color{blue},commentstyle=\color{darkgreen},stringstyle=\color{red},numberstyle=\tiny,numbersep=5pt,breaklines=true,showstringspaces=false,basicstyle=\footnotesize,emph={label}}

\renewcommand{\backref}[1]{}


%\numberwithin{equation}{section} \numberwithin{figure}{section}
%\numberwithin{table}{section}


\newenvironment{sistema}%
{\left\lbrace\begin{array}{@{}l@{}}}%
{\end{array}\right.}

\newcommand{\de}{\mathrm{d}}
\newcommand{\specialfootnote}[1]{%
  \begingroup
    \def\thefootnote{\textdagger}\footnote{#1}%
    \addtocounter{footnote}{-1}%
  \endgroup}

\newsavebox{\mybox}

\newtheorem{theorem}{Theorem}[section]
\newtheorem{lemma}[theorem]{Lemma}
\newtheorem{proposition}[theorem]{Proposition}
\newtheorem{corollary}[theorem]{Corollary}

%\newenvironment{proof}[1][Proof]{\begin{trivlist}
%\item[\hskip \labelsep {\bfseries #1}]}{\end{trivlist}}
\newenvironment{definition}[1][Definition]{\begin{trivlist}
\item[\hskip \labelsep {\bfseries #1}]}{\end{trivlist}}
\newenvironment{example}[1][Example]{\begin{trivlist}
\item[\hskip \labelsep {\bfseries #1}]}{\end{trivlist}}
\newenvironment{remark}[1][Remark]{\begin{trivlist}
\item[\hskip \labelsep {\bfseries #1}]}{\end{trivlist}}

%\newcommand{\qed}{\nobreak \ifvmode \relax \else
%      \ifdim\lastskip<1.5em \hskip-\lastskip
%      \hskip1.5em plus0em minus0.5em \fi \nobreak
%      \vrule height0.75em width0.5em depth0.25em\fi}




\newcommand{\Rey}{\textup{Re}}
\newcommand{\St}{\textup{St}}
\newcommand{\Pra}{\textup{Pr}}

\newcommand{\ddt}{\partial_t}
\newcommand{\Dt}{\textup{D}_t}
\newcommand{\Dpt}{\textup{D}_{p,t}}
\newcommand{\ddi}{\partial_i}
\newcommand{\ddj}{\partial_j}
\newcommand{\ddk}{\partial_k}
\newcommand{\Div}{\nabla\cdot}

\newcommand{\Ue}{U_\epsilon}
\newcommand{\Ta}{T_\alpha}
\newcommand{\Tb}{T_\beta}
\newcommand{\Ca}{C_\alpha}
\newcommand{\Cb}{C_\beta}


%\graphicspath{{./grafici/},{./foto/}}

\begin{document}
\maketitle 
\begin{abstract}
 {\tt reactingTwoPhaseEulerFoam} is a segregated semi-implicit solver for a mixture of two interpenetrating continuum phases, characterized by their own volume fraction, velocity and energy/enthalpy fields, with phase change. In the following, the model equations are derived and their implementation is shown .\\

 {\em Note}: this documentation is not approved not endorsed by the OpenFOAM Foundation or by ESI Ltd, the owner of OpenFOAM\textsuperscript{\textregistered}.
\end{abstract}


\section{Pressure equation}
In a two-phase system the discretized momentum equations for phase 1 and 2 can be written as

\begin{equation} \label{eq1}
\left[ U_p \right]_1= \left[ \frac{H}{a_p} \right]_1 - \left[ \frac{\alpha_1}{a_p}{\nabla(p)} \right]_1
\end{equation}
\begin{equation}\label{eq2}
\left[ U_p \right]_2= \left[ \frac{H}{a_p} \right]_2 - \left[ \frac{\alpha_1}{a_p}{\nabla(p)} \right]_2\\
\end{equation}

where $a_p$ indicates the discretization coefficients of the vector velocity $U$ in the center $p$ of the cell and $H$ includes the source terms and the terms depending on the velocity $U$ in the neighbouring cells.

 Continuity equations of the two phases can be rearranged to obtain, for phase 1,
 
\begin{eqnarray}\label{eq3}
\frac{\partial \alpha_1 \rho_1}{\partial t} + \nabla \cdot \left( \alpha_1 \rho_1 U_1 \right) = \Gamma_{21}^{tot} \nonumber\\
\frac{\alpha_1}{\rho_1}\frac{\partial \rho_1}{\partial t} + \frac{\partial \alpha_1}{\partial t} + \nabla \cdot \left( \alpha_1 U_1 \right) + \frac{\alpha_1 U_1}{\rho_1} \cdot \nabla \left( \rho_1 \right )= \frac{\Gamma_{21}^{tot}}{\rho_1} \nonumber\\
\nabla \cdot \left( \alpha_1 U_1 \right) = - \frac{\partial \alpha_1}{\partial t} + \frac{\Gamma_{21}^{tot}}{\rho_1} - \frac{\alpha_1}{\rho_1} \frac{d\rho_1}{dt}.
\end{eqnarray}
and for phase $2$ 
\begin{equation}\label{eq4}
\nabla \cdot \left( \alpha_2 U_2 \right) = - \frac{\partial \alpha_2}{\partial t} + \frac{\Gamma_{12}^{tot}}{\rho_2} - \frac{\alpha_2}{\rho_2} \frac{d\rho_2}{dt}.
\end{equation}

The pressure equation is obtained by multiplying Eq. (\ref{eq1}) and (\ref{eq2}) by $\alpha_1$ and $\alpha_2$, by taking the divergence of their sum and combining with Eq. (\ref{eq3}) and ({\ref{eq4})

\begin{eqnarray}\label{eq5}
\frac{\alpha_1}{\rho_1} \frac{d\rho_1}{dt} + \frac{\alpha_2}{\rho_2} \frac{d\rho_2}{dt} + \nabla \cdot \left[ \frac{\alpha_1 H}{a_p} \right]_1 + \nabla \cdot \left[ \frac{\alpha_2 H}{a_p} \right]_2 +\nonumber\\
-\nabla \cdot \left[ \frac{\alpha_1 \alpha_1}{a_p}\nabla(p) \right]_1 - \nabla \cdot \left[ \frac{\alpha_2 \alpha_2}{a_p}\nabla(p)  \right]_2  = \frac{\Gamma_{21}^{tot}}{\rho_1} + \frac{\Gamma_{12}^{tot}}{\rho_2}.
\end{eqnarray}

%
% % % % % % % % % % % % % % % % % % % % % % % % % % % % % % % % %
\subsection{PEqn.H}

Equation (\ref{eq5}) is implemented as 

\begin{verbatim}
 pEqnIncomp + pEqnComp1 + pEqnComp2 = 0.
\end{verbatim}

\begin{lstlisting}[frame=single]  
        fvScalarMatrix pEqnIncomp
        (
            fvc::div(phiHbyA)
          - fvm::laplacian(rAUf, p_rgh)
        );

        {
            fvScalarMatrix pEqn(pEqnIncomp);

            if (pEqnComp1.valid())
            {
                pEqn += pEqnComp1();
            }

            if (pEqnComp2.valid())
            {
                pEqn += pEqnComp2();
            }

            solve
            (
                pEqn,
                mesh.solver(p_rgh.select(pimple.finalInnerIter()))
            );
\end{lstlisting}\label{lst1}

 {\tt pEqnComp1} and {\tt pEqnComp2} are given, respectively, by
 
 \begin{equation}\label{eq1c}
\frac{1}{\rho_1}\left[ \frac{\partial}{\partial t}(\alpha_1 \rho_1) +\nabla \cdot\left({\alpha_1 \rho_1 U_1}\right)- \rho_1 \frac{\partial}{\partial t}(\alpha_1) - \rho_1 \nabla \cdot(\alpha_1 U_1) \right]
\end{equation}

\begin{lstlisting}[frame=single] 

            pEqnComp1 =
            (
                phase1.continuityError()
              - fvc::Sp(fvc::ddt(alpha1) + fvc::div(alphaPhi1), rho1)
            )/rho1
          + (alpha1*psi1/rho1)*correction(fvm::ddt(p_rgh));

\end{lstlisting}\label{lst2}

 \begin{equation}\label{eq2c}
\frac{1}{\rho_2}\left[ \frac{\partial}{\partial t}(\alpha_2 \rho_2) +\nabla \cdot\left({\alpha_2 \rho_2 U_2}\right)- \rho_2 \frac{\partial}{\partial t}(\alpha_2) - \rho_2 \nabla \cdot(\alpha_2 U_2) \right]
\end{equation}

\begin{lstlisting}[frame=single] 

            pEqnComp2 =
            (
                phase2.continuityError()
              - fvc::Sp(fvc::ddt(alpha2) + fvc::div(alphaPhi2), rho2)
            )/rho2
          + (alpha2*psi2/rho2)*correction(fvm::ddt(p_rgh));

\end{lstlisting}\label{lst2}

where {\tt phase1.continuityError} and {\tt phase2.continuityError} represent the first two terms in (\ref{eq1c}) and (\ref{eq2c}) defined in movingPhaseModel.C 

\begin{lstlisting}[frame=single] 
 continuityError_ =
        fvc::ddt(*this, rho) + fvc::div(alphaRhoPhi_)
       - (this->fluid().fvOptions()(*this, rho)&rho);
\end{lstlisting}\label{lst3}

Recalling the definition of lagrangian derivative, (\ref{eq1c}) and (\ref{eq2c}) can be written as

\begin{equation}\label{eq3c}
\frac{\alpha_1}{\rho_1} \frac{d\rho_1}{dt},\\
\frac{\alpha_2}{\rho_2} \frac{d\rho_2}{dt}
\end{equation}

that represent the first two terms in the pressure equation (5) .

A correction term is added to {\tt pEqnComp1} and {\tt pEqnComp2} in listing \ref{lst2} 

\begin{equation}\label{eq4c}
\frac{\alpha_1 \psi_1} {\rho_1}\left[ \frac{dp}{dt} - \left( \frac{dp}{dt} \right)_{old} \right], \\
\frac{\alpha_2 \psi_2} {\rho_2}\left[ \frac{dp}{dt} - \left( \frac{dp}{dt} \right)_{old} \right]
\end{equation}

\begin{lstlisting}[frame=single] 

          (alpha1*psi1/rho1)*correction(fvm::ddt(p_rgh));

          [...]
          
          (alpha2*psi2/rho2)*correction(fvm::ddt(p_rgh));
          
\end{lstlisting}\label{lst4}

Phase change terms in the r.h.s. of Eq. (\ref{eq5}) ($\frac{\Gamma_{21}^{tot}}{\rho_1} + \frac{\Gamma_{12}^{tot}}{\rho_2}$) are  added to {\tt pEqnComp1} and {\tt pEqnComp2}

\begin{lstlisting}[frame=single] 
    if (fluid.transfersMass())
    {
        if (pEqnComp1.valid())
        {
            pEqnComp1.ref() -= fluid.dmdt()/rho1;
        }
        else
        {
            pEqnComp1 = fvm::Su(-fluid.dmdt()/rho1, p_rgh);
        }

        if (pEqnComp2.valid())
        {
            pEqnComp2.ref() += fluid.dmdt()/rho2;
        }
        else
        {
            pEqnComp2 = fvm::Su(fluid.dmdt()/rho2, p_rgh);
        }
    }
\end{lstlisting}\label{lst6}

where

\begin{equation}\label{eq6c}
{\tt fluid.dmdt()} = \Gamma_{21}^{tot}, \\
{\tt -fluid.dmdt()} = \Gamma_{12}^{tot}
\end{equation}

 {\tt pEqnIncomp} is given by
 
 \begin{equation}\label{eq7c}
 \nabla \cdot \left[ {\tt phiHbyA} \right] - \nabla \cdot \left[ {\tt rAUf} \nabla \left(p\right) \right]
 \end{equation}
 
 \begin{lstlisting}[frame=single] 
         fvScalarMatrix pEqnIncomp
        (
            fvc::div(phiHbyA)
          - fvm::laplacian(rAUf, p_rgh)
        );
 \end{lstlisting} \label{lst7}

where {\tt phiHbyA} is the weighted sum of the difference between the predicted fluxes ({\tt phiHbyA1} and {\tt phiHbyA2}) and the buoyancy fluxes ({\tt phig1} and {\tt phig2}) 

\begin{equation}\label{eq8c}
{\tt phiHbyA} = \alpha_1({\tt phiHbyA1} - {\tt phig1}) + \alpha_2({\tt phiHbyA2} - {\tt phig2})
\end{equation}

 \begin{lstlisting}[frame=single] 
    surfaceScalarField phiHbyA
    (
        "phiHbyA",
        alphaf1*(phiHbyA1 - phig1) + alphaf2*(phiHbyA2 - phig2)
    );
 \end{lstlisting} \label{lst8}
 
 The predicted flux for phase 1 is given by
 
\begin{equation}\label{eq9c}
{\tt phiHbyA1} = \frac{1}{\frac{\alpha_1 \rho_1 + Vmf}{\Delta t} + a_{p,1} + Kdf}\left[ \frac{\alpha_1 \rho_1 + Vmf}{\Delta t}U_{old} + H_1 + Vmf\frac{dU_2}{dt} + KdfU_2 \right]
\end{equation} 

\begin{lstlisting}[frame=single] 
    phiHbyA1 =
        rAUf1
       *(
            (alphaRhof10 + Vmf)
           *MRF.absolute(phi1.oldTime())/runTime.deltaT()
          + fvc::flux(U1Eqn.H())
          + Vmf*ddtPhi2
          + Kdf*MRF.absolute(phi2)
          - Ff1()
        );
\end{lstlisting} \label{lst9}

\begin{lstlisting}[frame=single] 
surfaceScalarField rAUf1
(
    IOobject::groupName("rAUf", phase1.name()),
    1.0
   /(
        (alphaRhof10 + Vmf)/runTime.deltaT()
      + fvc::interpolate(U1Eqn.A())
      + Kdf
   )
);
\end{lstlisting} \label{lst10}

{\tt Vmf} and {\tt Kdf} are, respectively, virtual mass and drag coefficients. Since the terms in the denominator of {\tt rAUf1} multiply the vector velocity $U$, putting together the momentum transfer terms in Eq. (\ref{eq9c}) we obtain, for phase 1, 

\begin{eqnarray}
\frac{\alpha_1 \rho_1 + Vmf}{\Delta t}U_1 - \frac{\alpha_1 \rho_1 + Vmf}{\Delta t}U_{1}^{old}  + Vmf\frac{\partial U_2}{\partial t} + Kdf U_1 - Kdf U_2 = ...\\
...Vmf\left( \frac{\partial U_1}{\partial t} - \frac{\partial U_2}{\partial t} \right) + Kdf\left(U_1 - U_2\right) + \alpha_1 \rho_1 \frac{\partial U_1}{\partial t}
\end{eqnarray}

Recalling that in the {\tt UEqn } the following terms have been added to {\tt U1Eqn.H()}

\begin{equation}
Vmf \left[ 2 U_1 \nabla \cdot U_1 + \frac{\partial U_1}{\partial t} \right] - Vmf \left[ 2 U_2 \nabla \cdot U_2 + \frac{\partial U_2}{\partial t} \right]
\end{equation}

\begin{lstlisting}[frame=single] 

    fvVectorMatrix UgradU1
    (
        fvm::div(phi1, U1) - fvm::Sp(fvc::div(phi1), U1)
      + MRF.DDt(U1)
    );

    fvVectorMatrix UgradU2
    (
        fvm::div(phi2, U2) - fvm::Sp(fvc::div(phi2), U2)
      + MRF.DDt(U2)
    );
    
    [...]
    
    {
        U1Eqn =
        (
            fvm::div(alphaRhoPhi1, U1) - fvm::Sp(fvc::div(alphaRhoPhi1), U1)
          + fvm::Sp(dmdt12, U1) - dmdt12*U2
          + MRF.DDt(alpha1*rho1, U1)
          + phase1.turbulence().divDevRhoReff(U1)
          + Vm*(UgradU1 - (UgradU2 & U2))            
        );
        [...]
    }

\end{lstlisting}

and using the lagrangian derivatives, it results that the virtual mass term in {\tt PEqn } is given by

\begin{equation}
2 Vmf \left[ \frac{d U_1}{dt} - \frac{d U_2}{dt} \right].
\end{equation}

Including implicit part of the momentum transfer terms in the coefficients $a_p$ and the explixit part in $H$, the predicted fluxes can be written as
\begin{equation}
 \left[ \frac{\alpha_1 H}{a_p} \right]_1, \\
 \left[ \frac{\alpha_2 H}{a_p} \right]_2
\end{equation}

Since we are solving for $p - \rho g h$, the buoyancy flux {\tt phig} in (\ref{lst8}) has to be removed 
\begin{lstlisting}[frame=single] 

    surfaceScalarField ghSnGradRho
    (
        "ghSnGradRho",
        ghf*fvc::snGrad(rho)*mesh.magSf()
    );
    
    surfaceScalarField phig1
    (
        IOobject::groupName("phig", phase1.name()),
        alpharAUf1
       *(
            ghSnGradRho
          - alphaf2*(rhof1 - rhof2)*(g & mesh.Sf())
        )
    );


    surfaceScalarField phig2
    (
        IOobject::groupName("phig", phase2.name()),
        alpharAUf2
       *(
            ghSnGradRho
          - alphaf1*(rhof2 - rhof1)*(g & mesh.Sf())
        )
    );
    
\end{lstlisting}

Finally, {\tt pEqnIncomp} in Eq. (\ref{eq7c}) reads

\begin{equation}
\nabla \cdot \left[ \frac{\alpha_1 H}{a_p} \right]_1 + \nabla \cdot \left[ \frac{\alpha_2 H}{a_p} \right]_2 - \nabla \cdot \left[ \frac{\alpha_1 \alpha_1}{a_p}\nabla(p) \right]_1 - \nabla \cdot \left[ \frac{\alpha_2 \alpha_2}{a_p}\nabla(p)  \right]_2. 
\end{equation}

 After solving for the PEqn, fluxes, static pressure and density are updated. 
 
\begin{lstlisting}[frame=single] 
            phi = phiHbyA + pEqnIncomp.flux();

            surfaceScalarField phi1s
            (
                phiHbyA1
              + alpharAUf1*mSfGradp
              - rAUf1*Kdf*MRF.absolute(phi2)
            );

            surfaceScalarField phi2s
            (
                phiHbyA2
              + alpharAUf2*mSfGradp
              - rAUf2*Kdf*MRF.absolute(phi1)
            );

            surfaceScalarField phir
            (
                ((phi2s + rAUf2*Kdf*phi1s) - (phi1s + rAUf1*Kdf*phi2s))
               /(1.0 - rAUf1*rAUf2*sqr(Kdf))
            );

            phi1 = phi - alphaf2*phir;
            phi2 = phi + alphaf1*phir;
            
%\end{lstlisting} 

\begin{lstlisting}[frame=single] 

    p = max(p_rgh + rho*gh, pMin);

    p_rgh = p - rho*gh;

    rho1 += psi1*(p_rgh - p_rgh_0);
    rho2 += psi2*(p_rgh - p_rgh_0);

    rho = fluid.rho();
    p_rgh = p - rho*gh;

\end{lstlisting}  



\newpage
\appendix
%\section{Mathematical symbols}
%
%\begin{tabular}{ll}
%\toprule
%Symbol & meaning \\
%\midrule
%$\alpha $ & volume fraction\\
%$\rho  $ & density (obeys an equation of state)\\
%$e$       & specific internal energy (per unit of volume)\\
%$k$       & specific kinetic energy (per unit of volume)\\
%$h$       & specific enthalpy (per unit of volume)\\
%$T$       & temperature\\
%$p$       & Continuous phase pressure\\
%$p_D$     & dispersive pressure\\
%$\mathbf{U}$ & velocity vector\\
%$\mathbf{R^{\rm eff}}$ & Effective stress tensor (viscous + turbulent)\\
%$\mathbf{F}$   & Momentum transfer terms\\
%$\mathbf{S}$ & Deviatoric stress tensor \\
%$\mathbf q$  & Conductive heat flux\\
%$\mathbf{g}$ & gravitational acceleration\\
%$\nu_i$        & kinematic (first) viscosity coefficient\\
%$\lambda_i$    & kinematic (second) viscosity coefficient\\
%$\mathrm{Pr}$& (turbulent) Prandtl number \\
%$D_T$          & Dispersive (granular) energy flux coefficient\\
%$K$            & Momentum/Heat transport coefficients\\
%$C_p$          & Specific heat at constant pressure\\
%$C_v$          & Specific heat at constant volume\\
%\midrule
%Subscripts & \\
%\midrule
%1 & phase 1\\
%2 & phase 2\\
%$r$ & relative (velocity) \\
%$i$ & generic phase $i$ \\
%$d$ & Drag \\
%$l$ & Lift \\
%$vm$& Virtual Mass \\
%$wl$& Wall Lubrication \\
%$td$& Turbulent Diffusion \\
%$ht$& Conductive Heat Transfer\\
%\bottomrule
%\end{tabular}
%\newpage
%
%\section{Scalar and vector fields}
%\begin{tabular}{lll}
%\toprule
%{volScalarField} & Formula \\
%\midrule
%{\tt rAU1}         & $\displaystyle \frac{1}{a_{1p}}$\\[5mm]
%{\tt alpharAUf1}   & $\displaystyle \frac{\alpha_1}{a_{1p}}$ \\[5mm]
%{\tt} D            & $\displaystyle D_{td}$ \\
%{\tt}  & $\displaystyle $ \\
%{\tt}  & $\displaystyle $ \\
%{\tt}  & $\displaystyle $ \\
%{\tt}  & $\displaystyle $ \\
%{\tt}  & $\displaystyle $ \\
%{\tt}  & $\displaystyle $ \\
%\midrule
%{surfaceScalarField} & Formula \\
%\bottomrule
%\end{tabular}
%\newpage



\end{document}
